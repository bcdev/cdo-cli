\documentclass[
square
,pdftex
%,smallborder
]{refrep}
\usepackage{charter}
\usepackage[pdfborder=000,pdfhighlight=/N]{hyperref}
\usepackage{makeidx}
\usepackage{manfnt}

%%%%%%%%%%%%%%%%%%%%%%%%%%%%%%%%%%%%%%%%%%%%%%%%%%%%%%%%%%%%%%%%%%%%%%%%%
%
\title{CDO Command Line Interface User's Guide and Reference}

\author{Ralf Quast\\
  Brockmann Consult\\
  Max-Planck-Stra\ss e 2\\
  21502 Geesthacht}

\date{Version 1.0}
%
%%%%%%%%%%%%%%%%%%%%%%%%%%%%%%%%%%%%%%%%%%%%%%%%%%%%%%%%%%%%%%%%%%%%%%%%%

%%%%%%%%%%%%%%%%%%%%%%%%%%%%%%%%%%%%%%%%%%%%%%%%%%%%%%%%%%%%%%%%%%%%%%%%%
%
\newcommand{\bc}{\href{http://www.brockmann-consult.de/}
    {Brockmann Consult}}
\newcommand{\cdo}{\href{http://www.mpimet.mpg.de/cdo/}{CDO}}
\newcommand{\cdolong}{\href{http://www.mpimet.mpg.de/cdo/}
    {Climate Data Operators}}
\newcommand{\gpl}{\href{http://www.gnu.org/licenses/gpl-3.0.html}
    {GNU General Public License}}
\newcommand{\java}{\href{http://java.sun.com/}{Java}}
\newcommand{\mpimet}{\href{http://www.mpimet.mpg.de/}
    {Max Planck Institute for Meteorology}}
\newcommand{\javase}{\href{http://java.sun.com/javase/downloads/}
    {Java SE Downloads}}
\newcommand{\zmaw}{\href{http://www.zmaw.de/}{ZMAW}}

\renewcommand{\attentionsymbol}{\textdbend}
\makeatletter
\renewenvironment{theindex}
               {\if@twocolumn
                  \@restonecolfalse
                \else
                  \@restonecoltrue
                \fi
                \begin{fullpage}
                \let\twocolumn\REF@twocolumn
                \columnseprule \z@
                \columnsep 35\p@
                \if@pageperchapter
                   \setcounter{page}{1}
                   \ifnum \c@secnumdepth >\m@ne
                      \refstepcounter{chapter}%
                      \typeout{\@chapapp\space\thechapter.}%
                      \addcontentsline{toc}{chapter}
                         {\protect\numberline{\thechapter}\indexname}%
                   \else
                      \addcontentsline{toc}{chapter}{\indexname}%
                   \fi
                   \addtocontents{lof}{\protect\addvspace{10\p@}}%
                   \addtocontents{lot}{\protect\addvspace{10\p@}}%
                   \chapter*{\indexname}
                \else
                   \chapter*{\indexname}
                   \addcontentsline{toc}{chapter}{\indexname}%
                \fi
                \@mkboth{\indexname}%
                        {\indexname}%
                \parindent\z@
                \parskip\z@ \@plus .3\p@\relax
                \let\item\@idxitem}
               {\end{fullpage}\if@restonecol\onecolumn\else\clearpage\fi}
\makeatother
%
%%%%%%%%%%%%%%%%%%%%%%%%%%%%%%%%%%%%%%%%%%%%%%%%%%%%%%%%%%%%%%%%%%%%%%%%%

\pagestyle{headings}
\emergencystretch1em

\setcounter{tocdepth}{2}
\makeindex


%%%%%%%%%%%%%%%%%%%%%%%%%%%%%%%%%%%%%%%%%%%%%%%%%%%%%%%%%%%%%%%%%%%%%%%%%
%
\begin{document}

\begin{fullpage}
\maketitle
\end{fullpage}

\tableofcontents
\newpage

\chapter{Introduction}

The \cdo\ Command Line Interface (CLI) is a command line tool which, for a
certain set of tasks, provides a simplified usage of the \cdolong\ (CDO)
developed by the \mpimet.

The CDO-CLI has been developed by \bc\ under contract awarded by
\zmaw\ \href{http://www.mad.zmaw.de/}{M\&D} and is distributed under
the \gpl\ (GPL).

\paragraph{Installation}
\seealso{Chapter~\ref{installation}}
Besides a recent version of the \cdo\ software, running the CDO-CLI requires
the \java\ Runtime Environment 6 or later. Please consult your system
administrator if your system is not compliant. Installing and checking the
CDO-CLI is described in Chapter~\ref{installation}.

\paragraph{Configuration}
\seealso{Chapter~\ref{configuration}}
The behaviour of the CDO-CLI is configurable to a certain degree. The default
configuration very likely suits your needs. Configuring the CDO-CLI is described
in Chapter~\ref{configuration}.

\paragraph{Command Line Usage}
\seealso{Chapter~\ref{cli}}
The CDO-CLI provides a simple command line interface to the CDO software. The
usage of this command line interface is explicitly described in
Chapter~\ref{cli}.

\paragraph{Defining Tasks}
\seealso{Chapter~\ref{taskDef}}
The vendor supplied set of tasks can easily be extended by providing user
defined tasks. The definition of tasks is explicitly described in
Chapter~\ref{taskDef}.

\paragraph{Task Reference}
\seealso{Chapter~\ref{taskRef}}
The vendor supplied tasks supported by the CDO-CLI include the calculation of
time series statistics as well as the selection of land grid points, water grid
points, and regional subsets. The complete set of vendor supplied tasks is
described in Chapter~\ref{taskRef}.

\paragraph{Command Reference}
\seealso{Chapter~\ref{commandRef}}
The task definition language includes a certain set of commands, which is
explicitly described in Chapter~\ref{commandRef}.


\chapter{Installation}
\label{installation}

\section{System requirements}

\subsection{Java Runtime Environment (JRE)}
Running the CDO-CLI software requires \java\ Runtime Environment 6 or later. To
verify the version of the installed JRE open a terminal window and type:

\begin{verbatim}
    java -version
\end{verbatim}

The output of this command should be similar to:

\begin{verbatim}
    java version "1.6.0"
    Java(TM) SE Runtime Environment ...
    Java HotSpot(TM) Client VM ...
\end{verbatim}

The latest JRE for Linux, Solaris and Windows can be obtained from \javase. In
order to install the \java\ software, please follow the instructions given on
\javase\ or consult your system administrator.

\subsection{Climate Data Operators (CDO)}
Running the CDO-CLI software also requires \cdo\ version 1.0.9 or later. The
most recent version of this software can be obtained from the \mpimet. In order
to install the \cdo\ software, please follow the accompanying instructions or
consult your system administrator.

\section{Installation instructions}

\subsection{Installing}
For installing the CDO-CLI software simply unpack the ZIP archive 

\begin{verbatim}
    unzip cdo-cli-1.0-distribution.zip
\end{verbatim}

and move the resulting directory to your favourite location, for example: 

\begin{verbatim}
    mv cdo-cli-1.0 ${HOME}
\end{verbatim}

The CDO-CLI executables then reside in

\begin{verbatim}
    ${HOME}/cdo-cli-1.0/bin
\end{verbatim}

You probably want to make these executable generally accessible from the command
line by editing the PATH environment variable defined in your user profile, for
example:

\begin{verbatim}
    PATH="${PATH}:${HOME}/cdo-cli-1.0/bin"; export PATH
\end{verbatim}

In addition, you might have to customize the CDO-CLI configuration properties
defined in

\begin{verbatim}
    ${HOME}/cdo-cli-1.0/config.properties
\end{verbatim}

\attention
Note that the property defining the path of the \cdo\ executable must correspond
to a \cdo\ executable actually installed on your system. For example, define

\begin{verbatim}
    cdo-path                = ${user.home}/bin/cdo
\end{verbatim}

when the \cdo\ executable is installed in your home directory. All configuration
options are explained in detail in Chapter~\ref{configuration}.

\subsection{Checking}
For checking the CDO-CLI installation, open a terminal window and type:

\begin{verbatim}
    cdo-cli
\end{verbatim}

You should read the CDO-CLI usage message. Then type:
 
\begin{verbatim}
    cdo-cli-test
\end{verbatim}

When the CDO-CLI software is working properly, the resulting output will be
similar to:

\begin{verbatim}
    task subregion completed in 0.005 minutes
    task field-mean completed in 0.004 minutes
    task calendar-year-stat-mean completed in 0.018 minutes
    task calendar-year-stat-sum completed in 0.008 minutes
    task 10-year-stat-mean completed in 0.017 minutes
    task 30-year-stat-mean completed in 0.017 minutes
    task seasonal-stat-mean completed in 0.017 minutes
    task trend completed in 0.007 minutes
\end{verbatim}

You might also study the log file

\begin{verbatim}
    cdo-cli.log
\end{verbatim}

if you are interested in details. Note that it is expected that the log
report exhibits warnings. Severe errors, however, are not expected; any
occurrence of the word \emph{severe} in the log file clearly indicates
that the software does not run properly.

\chapter{Configuration}
\label{configuration}

The behaviour of the CDO-CLI is configurable to a certain degree. The
actual configuration is stored in the file

\begin{verbatim}
    config.properties
\end{verbatim}

which is located in the main directory of the CDO-CLI installation. There
are several name-value pairs defined in the configuration file, each of which
controls a certain behaviour.

\begin{description}

\item[cdo-path]\index{cdo-path}
The path (or a colon-separated list of paths) indicating where the \cdo\
executable is installed on your system. The preconfigured value is suited
for users at \zmaw.

\item[output-directory]\index{output-directory}
The directory path where all output stored, except for temporary files and
the log file; the latter is always written to the current working directory.
The preconfigured value corresponds to the current working directory:
 
\begin{verbatim}
    output-directory        = .
\end{verbatim}

\item[prompt]\index{prompt}
If set to \emph{true} the user is asked to either accept or reject input
files with an unexpected number of time steps. If set to \emph{false} such
files are accepted but a warning is logged. The preconfigured value is:

\begin{verbatim}
    prompt                  = false
\end{verbatim}

\item[log-level]\index{log-level}
Only messages corresponding to a level higher than or equal to the value of
this property are logged. The possible values of this property are \emph{off},
\emph{severe}, \emph{warning}, \emph{info}, \emph{fine}, \emph{finer},
\emph{finest}, and \emph{all}. The preconfigured value is:

\begin{verbatim}
    log-level               = info
\end{verbatim}

Note that the values for this property are not case sensitive.

\item[console-log-level]\index{console-log-level}
Only messages corresponding to a level higher than or equal to the value of
this property appear on the console. The possible values of this property are
\emph{off}, \emph{severe}, \emph{warning}, \emph{info}, \emph{fine},
\emph{finer}, \emph{finest}, and \emph{all}. The preconfigured value is:

\begin{verbatim}
    console-log-level       = info
\end{verbatim}

Note that a value corresponding to a level lower than the log level has no
effect.

\item[temp-dir-path]\index{temp-dir-path}
The directory path used for storing temporary files. The preconfigured value
corresponds to the directory where the system usually stores temporary files:
\begin{verbatim}
    temp-dir-path           = ${java.io.tmpdir}
\end{verbatim}

\item[temp-file-prefix]\index{temp-file-prefix}
The prefix prepended to the name of temporary files created by the CDO-CLI. The
preconfigured value is:
\begin{verbatim}
    temp-file-prefix        = cdo-cli
\end{verbatim}

\item[temp-file-suffix]\index{temp-file-suffix}
The suffix appended to the name of temporary files created by the CDO-CLI. The
preconfigured value is:
\begin{verbatim}
    temp-file-suffix        = .nc
\end{verbatim}

\item[delete-on-exit]\index{delete-on-exit}
If set to \emph{true} temporary files are deleted on exit of the \java\
Virtual Machine. If set to \emph{false} temporary files are not deleted.
The preconfigured value is:
\begin{verbatim}
    delete-on-exit          = true
\end{verbatim}

\end{description}

\attention
Note that there are further elements defined in configuration file, but only
those properties listed above are intended to be changed by clients.

\chapter{Command Line Usage}
\label{cli}

usage : cdo-cli [OPTIONS] [TASK [PARAMETERS]] [OPERANDS]


OPTIONS

    -c FILE
        the path of the configuration file to be loaded

    -d DIR
        the path of the output directory

    -o FILE
        the path or name of the output file

    -p
        the user is prompted for all (obligatory) task parameters


TASK

    The base name of one of the XML files in the 'task' directory


PARAMETERS

    istep=ISTEP
        the time step of the input time series to be processed

    ostep=OSTEP
        the time step of the output time series to be calculated

    box=WLON,ELON,SLAT,NLAT
        western longitude, eastern longitude, southern latitude, northern
        latitude for selecting a subregion

    land-water-mask=FILE
        the file path of the land-water mask for selecting either land or
        water points

    min-land-fraction=FRACTION
        any grid point with fractional land  coverage equal to or greater
        than this value is considered as as land point. The default value
        is 0.5

    max-land-fraction=FRACTION
        any grid point with fractional land coverage less than this value
        is considered as water point. The default value is 0.5


OPERANDS

    The input file paths


\chapter{Defining Tasks}
\label{taskDef}

\section{Properties}

\section{Parameters}

\section{Commands}


\chapter{Task Reference}
\label{taskRef}

\chapter{Command Reference}
\label{commandRef}

\printindex

\end{document}
%
%%%%%%%%%%%%%%%%%%%%%%%%%%%%%%%%%%%%%%%%%%%%%%%%%%%%%%%%%%%%%%%%%%%%%%%%%
